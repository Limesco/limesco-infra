\documentclass[a4paper,11pt]{memoir}

\beginperl
our $invoice = \%invoice;
our $account = \%account;
"";
\endperl

\usepackage[utf8]{inputenc}
\usepackage{fullpage}
\usepackage[fleqn]{amsmath}
\usepackage{hyperref}
\usepackage{xspace}
\usepackage{concmath}
\usepackage[T1]{fontenc}
\usepackage{tabularx}
\usepackage{longtable}

\newenvironment{nscenter}
 {\parskip=0pt\par\nopagebreak\centering}
 {\par\noindent\ignorespacesafterend}

\newenvironment{nospaceflalign*}
 {\setlength{\abovedisplayskip}{0pt}\setlength{\belowdisplayskip}{0pt}%
  \csname flalign*\endcsname}
 {\csname endflalign*\endcsname\ignorespacesafterend}

\newcommand{\totaalbedrag}{\footnotesize{Het totaalbedrag inclusief BTW dient te worden voldaan binnen 14 kalenderdagen na de
factuurdatum (\beginperl my ($y, $m, $d) = $invoice->{'date'} =~ /^(20\d\d)-(\d\d)-(\d\d)/; $OUT .= "$d-$m-$y"; \endperl).
Indien er vragen, opmerkingen of klachten zijn over deze factuur, kun je ons
een mail sturen via \href{mailto:support@limesco.nl}{support@limesco.nl} onder
vermelding van het factuurnummer (\beginperl $invoice->{'id'} \endperl).}}

\makepagestyle{factuur}
\makeevenfoot{factuur}{}{\totaalbedrag \\ Pagina \thepage}{}
\makeoddfoot{factuur}{}{\totaalbedrag \\ Pagina \thepage}{}

\mathindent=0pt

\begin{document}
\pagestyle{factuur}
\begin{flushright}
{\LARGE
Limesco B.V. \\
}
Toernooiveld 120 \\
6525 EC  Nijmegen \\
\end{flushright}

\begin{flushright}
\begin{tabular}{r l}
KVK-nummer & 55258778 \\
BTW-nummer & NL851628709B01 \\
IBAN & NL24RABO0169207587 \\
BIC  & RABONL2U \\
E-mailadres algemeen & \href{mailto:directie@limesco.nl}{directie@limesco.nl} \\
E-mailadres support & \href{mailto:support@limesco.nl}{support@limesco.nl} \\
\hline
Factuurnummer & \beginperl $invoice->{'id'} \endperl \\
Factuurdatum & \beginperl formatDate($invoice->{'date'}) \endperl
\end{tabular}
\end{flushright}

\begin{nospaceflalign*} \beginperl if( $account->{'company_name'} ) { $OUT .= '
\mbox{Bedrijf}       &: \mbox{'.$account->{'company_name'}.'}' } \endperl \\
\mbox{Naam}          &: \mbox{\beginperl $account->{'first_name'} . " " . $account->{'last_name'} \endperl} \\
\mbox{Adres}         &: \mbox{\beginperl $account->{'street_address'} \endperl} \\
\mbox{Postcode}      &: \mbox{\beginperl $account->{'postal_code'} \endperl} \\
\mbox{Plaats}        &: \mbox{\beginperl $account->{'city'} \endperl} \\
\end{nospaceflalign*}

\begin{nscenter}
\line(1,0){500}
\end{nscenter}

\newcolumntype{R}{>{\raggedleft\arraybackslash}X}
\newcolumntype{A}{>{\raggedleft\arraybackslash}p{3.0cm}}
\newcolumntype{B}{>{\raggedleft\arraybackslash}p{2.6cm}}
\newcolumntype{S}{>{\raggedleft\arraybackslash}p{3.0cm}}
\begin{longtable}{ | p{6.7cm} | B | S | S | }
\hline
\textbf{Omschrijving}      & \textbf{Aantal}  & \textbf{Stuksprijs} & \textbf{Bedrag} \\
\hline
\endfirsthead

\hline
\textbf{Omschrijving}      & \textbf{Aantal}  & \textbf{Stuksprijs} & \textbf{Bedrag} \\
\hline
\endhead

\beginperl
	foreach my $itemline (grep { $_->{'type'} ne "TAX" } @{$invoice->{'item_lines'}}) {
		my $service = $itemline->{'service'};
		my @description = split /\n/, $itemline->{'description'};
		if($itemline->{'type'} eq "NORMAL") {
			if($service eq "DATA") {
				$OUT .= sprintf("%s & %d KB & %.4f per MB & %.2f \\\\\n", shift @description, $itemline->{'item_count'}, $itemline->{'item_price'} * 1000,
					$itemline->{'rounded_total'});
			} else {
				$OUT .= sprintf("%s & %d & %.4f & %.2f \\\\\n", shift @description, $itemline->{'item_count'}, $itemline->{'item_price'},
					$itemline->{'rounded_total'});
			}
		} elsif($itemline->{'type'} eq "DURATION") {
			$OUT .= sprintf("%s & %d gesprekken & %.4f / gesprek & %.2f \\\\\n", shift @description, $itemline->{'number_of_calls'},
				$itemline->{'price_per_call'}, $itemline->{'rounded_total'});
			$OUT .= sprintf("%s & %d seconden & %.4f / minuut & \\\\\n", shift @description, $itemline->{'number_of_seconds'},
				$itemline->{'price_per_minute'});
		} else {
			die "Unknown item line type " . $itemline->{'type'} . "\n";
		}
		foreach my $line (@description) {
			$OUT .= "\\nopagebreak[4] $line & & & \\\\\n";
		}
		$OUT .= "\\hline\n";
	}
\endperl
\end{longtable}

\begin{flushright}
Totaal excl. BTW: \beginperl sprintf( "%.2f", $invoice->{'rounded_without_taxes'} ) \endperl
\end{flushright}

\begin{tabularx}{\textwidth}{ | X | X | R | }
\hline
Bedrag & BTW & BTW-bedrag \\
\hline
\beginperl
	foreach my $taxline (grep { $_->{'type'} eq "TAX" } @{$invoice->{'item_lines'}}) {
		$OUT .= sprintf("%.2f", $taxline->{'base_amount'}) . " & ";
		$OUT .= sprintf("%.1f\\%%", $taxline->{'taxrate'} * 100) . " & ";
		$OUT .= sprintf("%.2f", $taxline->{'rounded_total'}) . " \\\\ ";
	}
\endperl
\hline
\end{tabularx}

\begin{flushright}
Totaal incl. BTW: \textbf{\beginperl sprintf( "%.2f", $invoice->{'rounded_with_taxes'}) \endperl} \\
~~~~\\
Vorig saldo: \textbf{\beginperl sprintf( "%.2f", $balance) \endperl} \\
Factuurbedrag: \textbf{\beginperl sprintf( "%.2f", $invoice->{'rounded_with_taxes'}) \endperl} \\
\beginperl
	my $to_pay = $balance - $invoice->{'rounded_with_taxes'};
	$OUT .= sprintf("%s: \\textbf{%.2f}", $to_pay < 0 ? "Te betalen" : "Resterend saldo", $to_pay < 0 ? -$to_pay : $to_pay);
\endperl

\end{flushright}

\end{document}
